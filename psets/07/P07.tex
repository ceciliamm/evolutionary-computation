\documentclass[10pt, letterpaper]{article}
\usepackage[utf8]{inputenc}
\usepackage[spanish,es-nodecimaldot]{babel}
\usepackage[top=1in, bottom=1in, left=1in, right=1in]{geometry}

\usepackage{csquotes}

\usepackage{amsmath}
\usepackage{amsthm}
\usepackage{amssymb}

\theoremstyle{definition}
\newtheorem{definition}{Definición}[section]


\begin{document}

\begin{center}
    {\large \bfseries Práctica 7: Particle Swarm Optimization (PSO) y Evolución Diferencial (DE) \par}
    \vspace{0.2cm}
    Facultad de Ciencias, UNAM - Pablo A. Trinidad - 10 Mayo 2019
\end{center}

\begin{enumerate}
    \item \textbf{[Ejercicio teórico]} Identifica y menciona cuáles son ls
    componentes de búsqueda relacionados con la exploración y la explotación
    del espacio de búsqueda en el algoritmo de PSO y DE respectivamente.
        \begin{itemize}
            \item \textbf{PSO}: La componente de explotación de este algoritmo
                  está dado por el hecho de que la velocidad de una partícula
                  es alterada tomando en cuenta su mejor posición hasta el momento
                  mientras que la exploración se da por el hecho de que se toma
                  en cuenta la posición de los vecinos, así como la posición de
                  la partícula con mejor \textit{fitness} dentro de todo
                  el \textit{swarm}.
            \item \textbf{DE}: La componente de explotación de éste algoritmo
                  está dada por el hecho de que una nueva solución particular
                  de un agente únicamente se aceptará si mejora el \textit{fitness},
                  mientras que la componente de exploración únicamente sucede
                  al inicio cuando los agentes son ubicados de manera aleatoria.
        \end{itemize}
    \item \textbf{[Ejercicio teórico]} ¿Cómo modificaría el algoritmo de PSO
    canónico visto en clase para incluir aceleración?\\
        \textbf{Respuesta:} Utilizaría la noción de tiempo y la definición
        de aceleración ($v=at$) para alterar cada $v_d$ de una partícula. Si
        $p_d$ disminuye (con la noción del espacio de búsqueda), entonces la
        aceleración también disminuiría.
    \item \textbf{[Ejercicio teórico]} ¿Cuál es la diferencia entre la versión
    síncrona y la versión asíncrona del algoritmo PSO?
        \textbf{Respuesta:} Como \cite{pso-sync-vs-async} menciona, un algoritmo
        de optimización síncrono es aquel que requiere un
        \textit{punto de sincronización} al final de cada iteración para reunir
        los resultados y actualizar las variables de control, por el contrario,
        el método asíncrono no requiere de un \textit{punto de sincronización}
        para determinar la nueva dirección de búsqueda o los nuevos valores de
        las valores de los parámetros de control por lo que los cómputos y las
        partículas pueden ser distribuidos a través de múltiples recursos.
        Adicionalmente y en el contexto particular de PSO la síncrona
        actualiza las velocidades y posiciones de las partículas al final de cada
        iteración, mientras que la asíncrona que actualiza la posición y velocidad
        de las partículas continuamente basado en la información disponible.
    \item \textbf{[Ejercicio de programación]}
    \item \textbf{[Ejercicio teórico]}
    \item \textbf{[Ejercicio de programación]}
\end{enumerate}


% References
\clearpage
\begin{thebibliography}{9}
    \bibitem{hac-spso}
        Maurice Clerc.
        \textit{Standard Particle Swarm Optimisation.}
        2012. \texttt{hal-00764996}
    \bibitem{pso-sync-vs-async}
        Koh, B. I., George, A. D., Haftka, R. T., \& Fregly, B. J. (2006).
        \textit{Parallel asynchronous particle swarm optimization.}
        International journal for numerical methods in engineering,
        67(4), 578–595. doi:10.1002/nme.1646
\end{thebibliography}

\end{document}