\documentclass[10pt,letterpaper]{article}

\usepackage[utf8]{inputenc}
\usepackage[spanish,es-nodecimaldot]{babel}
\usepackage{amsmath}
\usepackage{amssymb}
\usepackage{graphicx}
\usepackage{mathtools}

\usepackage{multicol}

\usepackage{enumitem}

\usepackage[top=1in, bottom=1in, left=1in, right=1in]{geometry}

\renewcommand{\arraystretch}{1.5}

\begin{document}

\begin{titlepage}
    \centering

    {\scshape\LARGE Universidad Nacional Autónoma de México \par}

    \vspace{1cm}
    {\scshape\Large Facultad de Ciencias\par}
    \vspace{1.5cm}

    \begin{center}
        \includegraphics[scale=.1]{../../assets/img/logo.png}
    \end{center}

    \vspace{.8 cm}

    {\LARGE Tarea 04: \par}
    {\huge\bfseries Introducción al algoritmo genético \par}

    \vspace{0.5cm}
    \large{\itshape{Pablo A. Trinidad Paz}} \small{ - 419004279}

    \vfill

    Trabajo presentado como parte del curso de
    \textbf{Cómputo Evolutivo}
    impartido por el profesor \textbf{Mario Iván Jaen Márquez}. \par
    \vspace{0.5cm}
    Fecha de entrega: \textbf{Lunes 19 de Marzo de 2019}.
\end{titlepage}

\section{Planteamiento}
Implemente el algoritmo genético para que dados los siguientes ejercicios,
el algoritmo se aproxime a un individuo que cumpla las condiciones necesarias.

\begin{enumerate}
    \item Genere un cromosoma cuyas características indiquen a una persona con
    piel oscura, caballo negro, altura de 1.98 metros o cercana y complexidad
    delgada.
    \item Obtenga un cromosoma con las cualidades para ser luchador de sumo japonés
    \footnote{Investigue los valores promedio de un luchador de sumo y de un hombre
    japonés. Adjunte sus referencia}
    \item Elija un personaje de la serie que prefiera\footnote{Puede ser de cualquier
    tipo, ya sea un show de televisión, anime o serie animada, de preferencia use
    características poco comunes} y ajuste el algoritmo para obtener
    un cromosoma de características similares. Debe considerar por lo menos las
    siguientes características:
        \begin{itemize}
            \item Color de cabello
            \item Color de piel
            \item Color de ojos
            \item Altura
            \item Complexión
        \end{itemize}
    Adjunte un enlace a la referencia del personaje del cual se está tomando la información.

    La implementación del algoritmo debe cumplir lo siguiente:
        \begin{itemize}
            \item La población inicial debe ser aleatoria y de 100 individuos.
            \item Para la probabilidad de mutación utilice 0.1.
            \item Para el operador de cruza utilice el método de torneo en la
                  elección de padres y dos puntos de cruce (two point crossover).
            \item En la selección de individuos utilice ruleta.
            \item El \textit{fitness} se va definir a discreción cuidando que el
                  algoritmo garantice o se logre aproximar considerablemente a un individuo
                  que cumpla las características solicitadas. Se recomienda la siguiente estructura:\\[\baselineskip]
                  Sea $\mathcal{X}$ el individuo con las características deseadas y $x$ el individuo a evaluar.
                  La función $f$ de \textit{fitness} es:
                    \begin{equation*} \begin{split} \begin{gathered}
                        f = \sum_{i=0}^{n} \big( \mathcal{X}[i] - x[i] \big)^2
                    \end{gathered} \end{split} \end{equation*}
                  donde $n$ es el número de características.
            \item El cromosoma deberá considerar los valores de color de cabello, color
                  de ojos y color de piel usando 3 valores para cada característica donde para cada
                  valor el rango será entero $[0, 255]$; para el cabello y ojos será válido cualquier
                  color, cuide que el valor de la tez de piel se vea natural, asimismo considere
                  valores de altura coherentes (aproximadamente de $1$ a $2.20$ metros) y complexión,
                  se deja a discreción agregar más valores al cromosoma.
            \item Para graficar la imagen del individuo utilice la función indicada en el script
                  adjunto, y para cada ejercicio adjunte la imagen del individui que mejor cumpla
                  con los requisitos del ejercicio, su posible altura, complexión, el cromosoma, y
                  en su caso, las características adicionales.
        \end{itemize}
\end{enumerate}
\clearpage

\section{Solución}

\subsection{Implementación del algoritmo genético}
En ésta sección abordaremos las múltiples estrategias utilizadas durante la implementación
del algoritmo genético.

\subsubsection{Representación}

Con el objetivo de obtener una representación binaria del cromosoma se acordaron las siguientes
abstracciones de cada una de las características:

\begin{enumerate}
    \item \textbf{Color}: Representado por un valor RGB de tal forma que
          color $ \in \{0, 1,...,255\}^L, \; L=3$. Las características que usarán éste
          tipo de valor serán el color de piel, el color del cabello y el color de ojos.
    \item \textbf{Altura}: Medida en centímetros y válida en el rango $[0, 255]$. El rango
          utilizado ignora la recomendación con el objetivo de poder utilizar 8 bits para
          representarla.
    \item \textbf{Complexión}: Definida arbitrariamente en el rango $[0, 15]$ donde una
          complexión con valor $x=0$ se considera la complexión más pequeña posible y
          una complexión $x=15$ la más grande posible, es decir, la complexión crece de
          manera lineal de la más pequeña a la más grande conforme $x$ se crece. La
          complexión puede ser representada utilizando 4 bits.
\end{enumerate}

Dicho lo anterior podemos definir la representación binaria del cromosoma de un individuo
como la sucesión ordenada de dichas características de forma que los genes estén distribuidos
de la siguiente manera: \\

\begin{center} \begin{tabular}{ |c|c|c| }
    \hline
    \multicolumn{3}{|c|}{Distribución de genes dentro del cromosoma \scriptsize{(Indexado en 0)}} \\
    \hline
    Rango de bits & Característica & Rango en $\mathbb{Z}^+$\\
    \hline
    $[0, 23]$   & Color de cabello & $[0, 255]$ cada 8 bits \\
    $[24, 47]$  & Color de ojos & $[0, 255]$ cada 8 bits \\
    $[48, 71]$   & Color de piel & $[0, 255]$ cada 8 bits \\
    $[72, 79]$   & Altura & $[0, 255]$ \\
    $[80, 83]$   & Complexión & $[0, 15]$ \\
    \hline
\end{tabular} \end{center}

Adicionalmente, la codificación de cada gen será mediante Gray code con el objetivo de
reducir la distancia de Hamming.

\subsubsection{Función objetivo}

Gracias a que tenemos una representación binaria del cromosoma, que estamos utilizando
Gray code para la codificación de cada valor, y que queremos que nuestras poblaciones
asemejen lo más posible a un individuo dado, podemos establecer la función objetivo $f$
como la distancia de Hamming entre la cadena del individuo ideal y el individuo a evaluar.
Además, nótese que únicamente estaríamos realizando la decodificación al obtener un resultado aceptable.
Si quisiéramos que la función $f$ esté en el rango $[0, 1]$, entonces podemos definirla
como:

\begin{equation*} \begin{split} \begin{gathered}
    f(x) = 1 - \frac{h(\mathcal{X}, x)}{84}
\end{gathered} \end{split} \end{equation*}

Donde $\mathcal{X}$ es el individuo con características deseadas y $h(x, y)$ es la
función que calcula la distancia de Hamming entre dos cadenas $x$ y $y$.

\subsubsection{Inicialización}

La inicialización de la población consiste en la creación de cadenas binarias
aleatorias de longitud $L=84$. La población inicial tendrá 100 individuos.

\subsubsection{Mecanismo de selección de padres}

El método utilizado para la selección de de padres será Torneo ya que nos evitará
la necesidad de contar con una noción del \textit{fitness} promedio de la población.
La selección por torneo selecciona $\lambda$ mejores miembros de un subconjunto 
aleatorio de $\mu$ elementos hasta completar la cantidad deseado de padres $\rho$. \\

En nuestro caso particular deseamos obtener $\rho=6$ padres en cada iteración del
algoritmo genético y lo haremos a través de conjuntos de $\mu=33$ elementos escogiendo
$\lambda=2$ elementos en cada ronda del torneo con reemplazo.

\subsubsection{Operador de cruza}

El método utilizado será $n$-point crossover con $n=2$. En nuestro caso, las parejas
de cruzamiento serán unidas en el mismo orden en el que hayan sido seleccionadas durante
el torneo.

\subsubsection{Operador de mutación}

Tras la cruza, la mutación sucederá en los hijos resultantes invirtiendo cada bit
de su representación genotípica con probabilidad $p_m=0.1$.

\subsubsection{Selección de individuos}

La selección de individuos seguirá el método \textit{fitness proportional selection}
(también conocido como ruleta) usando \textbf{sigma scaling} descrito en \cite{iec}
para resolver los problemas de convergencia prematura. La nueva función objetivo
para \textbf{sigma scaling} será la siguiente
    \begin{equation*} \begin{split} \begin{gathered}
        f'(x) = \mathrm{max}(f(x) - (\bar{f} - c \cdot \sigma_f), 0 )
    \end{gathered} \end{split} \end{equation*}

Donde $f(x)$ es el fitness original del individuo, $\bar{f}$ es el promedio del fitness
de todos los individuos, $\sigma_f$ la desviación estándar y $c$ un valor constante,
en nuestro caso $c=2$.

El método \textit{FPS} asignará la probabilidad $P_{FPS}(i)$ a todos los individuos
y \textit{girará la ruleta} $N$ veces para obtener $N$ individuos. En nuetro caso
$N=100$ y
    \begin{equation*} \begin{split} \begin{gathered}
        P_{FPS}(i) = \frac{f'_i}{\sum_{j=1}^{N} f'_j}
    \end{gathered} \end{split} \end{equation*}

\clearpage
\subsection{Resultados}

\begin{enumerate}
    \item Ejercicio 1. \textbf{Valores del fenotipo y datos de ejecución:}
        \begin{center}\includegraphics[scale=0.5]{./assets/EX1Results.png}\end{center}
        \begin{center}\includegraphics[scale=1]{./assets/EX1Avatar.png}\end{center}
    \clearpage
    \item Ejercicio 2. Se consideró la complexión de tipo 15, color de piel (248, 230, 254)
          y color de cabello y ojos de (99, 50, 110) y la
          altura de 173 cm\footnote{https://en.wikipedia.org/wiki/Sumo\#Minimum\_height\_requirement}.
        \begin{center}\includegraphics[scale=0.5]{./assets/EX2Results.png}\end{center}
        \begin{center}\includegraphics[scale=1]{./assets/EX2Avatar.png}\end{center}
    \clearpage
    \item Ejercicio 3. Homer Simpson. Color de piel (254, 217, 15), color de cabello y ojos (202, 171, 119)
          complexión de tipo 10 y altura 183 cm \footnote{https://simpsons.fandom.com/wiki/Homer\_Simpson}
        \begin{center}\includegraphics[scale=0.5]{./assets/EX3Results.png}\end{center}
        \begin{center}\includegraphics[scale=1]{./assets/EX3Avatar.png}\end{center}
        \begin{center}\includegraphics[scale=0.1]{./assets/Homer.png}\end{center}
\end{enumerate}


% References
\clearpage
\begin{thebibliography}{9}

    \bibitem{iec}
      Eiben, A.E., Smith, J.E. (2015).
      \textit{Introduction to Evolutionary Computing}.
      Amsterdam, Netherlands: Springer 
      2nd edition. 287 p.
\end{thebibliography}

\end{document}