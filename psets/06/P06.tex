\documentclass[10pt,letterpaper]{article}

\usepackage[utf8]{inputenc}
\usepackage[spanish,es-nodecimaldot]{babel}
\usepackage{amsmath}
\usepackage{amssymb}
\usepackage{graphicx}
\usepackage{mathtools}

\usepackage{multicol}

\usepackage{enumitem}

\usepackage[top=1in, bottom=1in, left=1in, right=1in]{geometry}

\renewcommand{\arraystretch}{1.5}

\begin{document}

\begin{titlepage}
    \centering

    {\scshape\LARGE Universidad Nacional Autónoma de México \par}

    \vspace{1cm}
    {\scshape\Large Facultad de Ciencias\par}
    \vspace{1.5cm}

    \begin{center}
        \includegraphics[scale=.1]{../../assets/img/logo.png}
    \end{center}

    \vspace{.8 cm}

    {\LARGE Tarea 06: \par}
    {\huge\bfseries Estrategias evolutivas\par}

    \vspace{0.5cm}
    \large{\itshape{Pablo A. Trinidad Paz}} \small{ - 419004279}

    \vfill

    Trabajo presentado como parte del curso de
    \textbf{Cómputo Evolutivo}
    impartido por el profesor \textbf{Mario Iván Jaen Márquez}. \par
    \vspace{0.5cm}
    Fecha de entrega: \textbf{Jueves 4 de Abril de 2019}.
\end{titlepage}

\begin{enumerate}
    \item \textbf{[Ejercicio de programación]} Escribe una función que genere números
          pseudo-aleatorios de las distribución normal estándar $N(0, 1)$ a partir de
          números uniformemente distribuidos. Indica el método usado. \\[\baselineskip]

        \textbf{Solución:} Se implementó el método de muestreo de números pseudo-aleatorios
        descrito por Box-Muller\footnote{https://en.wikipedia.org/wiki/Box-Muller\_transform}.
        A continuación se presentan los resultados de la implementación comparados con el
        método $\mathrm{random.gauss(0, 1)}$ de la librería estándar de Python.
        \begin{center}
            \includegraphics[scale=.6]{./assets/ex1-results1.png}
        \end{center}
        \begin{center}
            \includegraphics[scale=.6]{./assets/ex1-results2.png}
        \end{center}
\end{enumerate}

\end{document}