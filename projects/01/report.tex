\documentclass[10pt, letterpaper]{article}
\usepackage[utf8]{inputenc}
\usepackage[spanish,es-nodecimaldot]{babel}
\usepackage[top=1in, bottom=1in, left=1in, right=1in]{geometry}

\usepackage{csquotes}

\usepackage{amsmath}
\usepackage{amsthm}
\usepackage{amssymb}

\theoremstyle{definition}
\newtheorem{definition}{Definición}[section]


\begin{document}

\begin{center}
    {\large \bfseries Estrategias evolutivas para 
    la obtención de la curva braquistócrona \par}
    \vspace{0.2cm}
    Facultad de Ciencias, UNAM - Pablo A. Trinidad - Mayo 2019
\end{center}

\begin{section}{Resumen}
    El problema de la curva de descenso más rápido plantea encontrar aquel
    camino en el plano vertical que lleve a un objeto deslizándose o girando
    sin fricción sobre él, de un punto $A$ a un punto $B$ que se encuentra
    estrictamente a una altura menor a la de $A$ y con una traslación horizontal,
    en la menor cantidad de tiempo posible. Aunque la solución analítica a ésta
    curva es conocida (curva braquistócrona), el objetivo de éste reporte es
    mostrar los resultados obtenidos al aplicar una estrategia evolutiva
    $(\mu + \lambda)$ a curvas discretizadas para resolver el problema
    mencionado anteriormente. El reporte incluye los detalles de representación
    de cada individuo dentro de la estrategia evolutiva, así como la función
    de costo que evalúa el desempeño de cada curva. La decisión de discretizar
    las curvas como colecciones de puntos surge del trabajo realizado por
    \textbf{Boschback} y \textbf{Dreckmann}\cite{es-brachistochrone} donde
    prueban su hipótesis de que una curva braquistócrona discretizada aproximada
    mediante estrategias evolutivas tendrá mejor desempeño que aquella, también
    discretizada, obtenida analíticamente.
\end{section}

\begin{section}{Introducción}
    El problema de la curva de descenso más rápido fue originalmente propuesto
    en 1696 por \textbf{Johann Bernoulli} aunque sus primeros planteamientos
    datan de fechas anteriores a esa\cite{3blue1brown}. El problema propuesto
    por Bernoulli dice lo siguiente:
    \begin{displayquote}
        Dados 2 puntos $A$ y $B$ en un plano vertical ¿cuál es la curva trazada
        por un punto que es afectado únicamente por la gravedad, el cuál inicia
        en $A$ y alcanza $B$ en la menor cantidad de tiempo?
    \end{displayquote}

    Y la solución encontrada por \textbf{Johann Bernoulli} se basa en uno de los
    principios de Fermat que enuncia que:
    \begin{displayquote}
        La ruta tomada por un rayo de luz entre dos puntos en el mismo medio
        siempre es la ruta que toma menor tiempo.
    \end{displayquote}

    Es así que a \textbf{Johann Bernoulli} se le ocurrió que podría modelar
    el problema de manera diferente: en lugar de pensar en él como un punto
    deslizándose a través de una curva, podría pensar en el problema como un
    rayo de luz viajando a través de diferentes medios con múltiples índices
    de refracción, implicando que el rayo de luz iría a diferentes velocidades
    conforme viajaba a través de los diferentes medios. Usando éste modelo
    Bernoulli notó que existía una relación constante entre la velocidad del
    rayo de luz y la distancia relativa a la altura del punto de origen $A$
    (Ley de Snell) y que podría obtener una solución final aproximando el
    problema a infinitas capas de diferentes materiales. Finalmente, J Bernoulli
    reconoció la relación de la ley de Snell con la ecuación diferencia de la
    curva de un Cicloide, encontrando así la solución analítica. \\

    De la analogía que Johan Bernoulli encuentra surge la inspiración de
    de plantear el problema como una secuencia discreta de diferentes segmentos
    de recta en las cuales un punto se deslizaría con diferentes velocidades.
    La inspiración fue posteriormente reforzada por el trabajo realizado por
    \textbf{Boschback} y \textbf{Dreckmann} donde demuestran por simulación que
    dada una definición discreta de una curva \textbf{braquistócrona aproximada (AB)},
    las soluciones encontradas por una estrategia evolutiva resultaban más eficientes
    que la aproximación analítica. Finalmente, y aunque la representación también
    era discretizada, se decidió no utilizar la técnica propuesta por
    \textbf{van Koot K.A.A.M. (KK)}\cite{vankoot} debido a que su solución dependía
    de la propuesta inicial de funciones a optimizar junto con análisis previo
    de dichas funciones alrededor del punto $A$.

    A continuación se muestra la similitud entre la función de tiempo dada por
    el análisis analítico del problema:
    \begin{equation} \begin{gathered}
        t = \int_{x_A}^{x^B} \sqrt{\frac{1 + (\frac{dy}{dx})^2}{2gy}} dx,
    \end{gathered} \end{equation}
    y la función de tiempo dada por la solución discretizada en $n$ puntos:
    \begin{equation} \begin{gathered}
        t = \sum_{i=0}^{n-1} \sqrt{\frac{1 + (\frac{y_{i+1} - y_i}{\Delta x})^2}{g(y_{i+1} + y_i)}}\Delta x
    \end{gathered} \end{equation}

    Debido al interés personal de poder visualizar como se comporta una estrategia
    evolutiva se optó por utilizar la representación de la curva braquistócrona
    aproximada (AB). Dicho lo anterior, podemos definir AB como:
    \begin{definition}{Approximated Brachistochrone (AB)}
        Sea $X$ la distancia horizontal entre los puntos $A$ y $B$ dividido
        en $n$ distancias iguales tal que:
        \begin{equation} \begin{gathered}
            x_0 = 0, x_1 = \frac{X}{n}, x_2 = \frac{2X}{n}, ..., x_n=\frac{nX}{n}=X
        \end{gathered} \end{equation}
        La curva braquistócrona aproximada (AB) para dos puntos $A$ y $B$ y un
        número $n \in \mathbb{N}, n \geq 1$, está definida como el conjunto de
        $n-1$ puntos con una altura aproximada $y_i$ en cada punto $x_i$ para
        $i \in \{1,...,n-1\}$ $P_i(x_i, y_i)$, con la posición de los puntos $y_0 := A_{vert}$
        y $y_n=0$.
        Finalmente definimos (AB) aproximando la curva generada por la unión
        mediante líneas rectas de los puntos vecinos.
    \end{definition}

    Ésta definición será utilizada posteriormente para representar y evaluar
    las soluciones que la estrategia evolutiva vaya generado.

\end{section}

\begin{section}{Estrategia evolutiva $(\mu + \lambda)$}
    El algoritmo utilizado para la solución de problema es la
    estrategia evolutiva ($\mu + \lambda)$,  ES-$(\mu + \lambda)$. En éste caso
    particular se utilizó una variable de control $\sigma$ inicializada en $1.0$
    por cada individuo que formará parte de su cromosoma y consecuentemente será
    modificada durante cada iteración del ES-$(\mu + \lambda)$.\\

    Los criterios de paro utilizados fueron un máximo de iteraciones $I$ y un
    valor $\epsilon$ que representa la precisión mínima que algoritmo aceptará
    como diferencia entre el valor promedio del fitness de la población anterior
    y el fitness promedio de la población actual.\\

    El algoritmo comienza con la generación aleatoria de individuos y tras cada
    iteración sucederá lo siguiente:
    \begin{enumerate}
        \item \textbf{Selección de padres:} Si $\lambda > \mu$, después de $\mu$
              padres seleccionados secuencialmente se seleccionarán nuevos padres
              de manera aleatoria, en el caso contrario, se seleccionarán $\lambda$
              hijos de manera secuencial.
        \item \textbf{Creación de hijos:} Todos los hijos heredan las mismas
              características que sus padres, en otras palabras, un hijo tiene
              un sólo padre y es una réplica exacta del padre antes de la mutación.
        \item \textbf{Mutación de los hijos:} Por cada gen $g_t$ en los cromosomas se
              escoge un valor aleatorio $z$ de una distribución normal utilizando
              $\mu=0$ y el valor de $\sigma$ del padre, y el gen resultante $g_{t+1}$
              resulta en $g_{t+1} = g_t + z$ siempre y cuando no exceda el espacio
              de búsqueda del algoritmo (definido en la siguiente sección). Al
              finalizar la mutación de los genes el valor $\sigma$ es alterado
              utilizando la expresión
              \begin{equation} \begin{gathered}
                \sigma_{t+1} = \sigma_t\exp\Big( \frac{z_1^*}{\sqrt{2N}} + \frac{z_2^*}{\sqrt{2\sqrt{N}}}\Big)
              \end{gathered} \end{equation}
              Donde $N$ es el número de cromosomas y $z_1^*$ y $z_2^*$ son valores
              aleatorios extraidos de una distribución normal con $\mu=0, \sigma=1$.
        \item \textbf{Selección de individuos:} Los $\mu$ individuos con mejor
              fitness son escogidos.
    \end{enumerate}
    \begin{subsection}{Codificación}
        La codificación de cada individuo resulta sencilla debido a la
        discretización mencionada en la definición de (AB). Para una AB particular
        en los puntos $A$ y $B$ y un $n$ específico, el cromosoma de un individuo
        es el vector $a = \{a_1, a_2, ..., a_{n-1}, \sigma_a\}$ donde $\sigma_a$
        es la desviación estándar utilizada como parámetro de control y cada
        $a_i, i=\{i,...,n-1\}$, representa un valor $y_i$ de la definición de $AB$
        para $P_i(x_i, y_i)$. Es importante mencionar que $a_i \in [S, A_{vert}]$
        para un $S_i$ arbitrario $\leq 0$. Los valores $x_i$ del punto $P_i$ son los descritos
        en la \textbf{Definición 2.1.}
    \end{subsection}
    \begin{subsection}{Función objetivo}
        La función objetivo es producto de un análisis sencillo de movimiento
        uniformemente acelerado sobre una recta y es el resultado de la suma
        de los tiempos calculados en cada segmento de recta, la expresión final
        resulta en:
        \begin{equation} \begin{gathered}
            F(\vec{y}) = \sqrt{\frac{2}{g}} \sum_{i=0}^{n-1}
                \frac
                    {\sqrt{(x_{i+1} - x_{i})^2 + (y_{i+1} - y_i)^2}}
                    {\sqrt{y_0 - y_i} + \sqrt{y_0 - y_{i+1}}}
        \end{gathered} \end{equation}
        Donde $\vec{y}$ es el conjunto de puntos $P_i(x_i, y_i)$ descrito en la
        \textbf{Definición 2.1}.\\

        Debido a que la expresión $\sqrt{\frac{2}{g}}$ únicamente escala el
        resultado final y no nos interesa conocer los tiempo exactos de la curva
        en la gravedad terrestre, dicho término fue ignorado resultando en:
        \begin{equation} \begin{gathered}
            F(\vec{y}) = \sum_{i=0}^{n-1}
                \frac
                    {\sqrt{(x_{i+1} - x_{i})^2 + (y_{i+1} - y_i)^2}}
                    {\sqrt{y_0 - y_i} + \sqrt{y_0 - y_{i+1}}}
        \end{gathered} \end{equation}
    \end{subsection}
\end{section}

\begin{section}{Resultados}
    Utilizando los valores propuestos por \textbf{Boschback} y \textbf{Dreckmann}
    de $\mu=10$ y $\lambda=100$ junto con una precisión $\epsilon=10^{-4}$,
    número máximo de iteraciones $I=10^5$, los puntos $A=(0, 10)$ y $B=(30, 0)$
    y $S_i = -15$ se obtuvieron los siguientes resultados:

    \begin{center} \begin{tabular}{ |c|c|c| }
        \hline
        $n - 1$ & generaciones & fitness promedio \\
        \hline
        10  & 138 & 7.2053 \\
        30  & 287 & 7.3715 \\
        50  & 630 & 7.3230 \\
        100  & 3090 & 8.2103 \\
        \hline
    \end{tabular} \end{center}

    \textbf{Evolución de las curvas:}

    Notamos que las curvas con $n \geq 30$ comienzan a estancarse en óptimos
    locales mientras que para $n$ pequeñas las curvas convergen relativamente
    rápido.
\end{section}

\begin{section}{Conclusiones}
    El algoritmo converge relativamente rápido (en cuestión de segundos) y las
    curvas generadas son extremadamente similares a aquellas generadas por
    un cicloide.

    \begin{subsection}{Mejoras}
        \textbf{Boschback} y \textbf{Dreckmann} proponen la definición analítica
        de la curva braquistócrona aproximada (CS, Cycloid Solution) de la serie
        de puntos $P_i(x_i, y_i)$ para $i={0, 1, ..., n}$. En su trabajo ellos
        demuestran a través de múltiples simulaciones de Monte Carlo que las
        soluciones encontradas con ES-$(\mu + \lambda)$ superan en rendimiento
        a aquellas soluciones obtenidas analíticamente (CS). Me gustaría poder
        replicar esa demostración.\\

        Referente al trabajo de \textbf{van Koot K.A.A.M. (KK)}, debido a que
        la representación del problema que propone depende de funciones
        analizadas alrededor del punto $A$ es posible obtener una serie de puntos
        de la forma $P_i(x_i, y_i)$ de su respectiva AB. Por lo tanto, me gustaría
        mucho poder implementar la representación de van Koot, transformar las
        soluciones obtenidas al tipo AB y finalmente comparar los resultados
        obtenidos.

        Por último, me quedé con la duda de cómo implementar ES-$(\mu + \lambda)$
        si el sistema estuviera corriendo en una simulación física, es decir,
        si existiera una representación gráfica que mostrara las soluciones
        y un motor físico que pudiera calcular los costos de cada una de las
        curvas, así como el impacto que dicha simulación tendría sobre el desempeño
        del algoritmo, aunque es claro, que ya no estaríamos hablando del mismo
        problema debido a las leyes del motor físico y también tendríamos que
        pensar en alguna otra forma de representar las soluciones que sea más
        placentera visualmente.
    \end{subsection}
\end{section}


% References
\clearpage
\begin{thebibliography}{9}

    \bibitem{es-brachistochrone}
      Borschbach, M., Dreckmann, W. (2007).
      \textit{On the Role of an Evolutionary Solution for the Brachistochrone-Problem}.
      University of Münster, Germany.
      IEEE - 1-4244-1340-0

    \bibitem{3blue1brown}
        3Blue1Brown (2006), Canal de YouTube.com
        \textit{The Brachistochrone, with Steven Strogatz}
        \small{https://www.youtube.com/watch?v=Cld0p3a43fU}

    \bibitem{vankoot}
        van Koot, K.A.A.M. (2014)
        \textit{Particle swarm optimization algorithms, analysis and computing - Master thesis}
        Eindhoven University of Technology, Netherlands.
\end{thebibliography}

\end{document}